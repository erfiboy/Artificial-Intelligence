\documentclass[a4paper,12pt]{article}
\usepackage{heading}
\usepackage{listings}
\usepackage[shortlabels]{enumitem}
\usepackage{hyperref}
\usepackage{float}
\usepackage{tikz}
\usepackage{amsmath}
\usepackage{amssymb}	
\usepackage{tcolorbox}
\usepackage{xepersian}
\usepackage{changepage}

\settextfont{XB Niloofar.ttf}
\usetikzlibrary{arrows}

\doublespacing


\begin{document}

\handout

{شبکه عصبی پرسپترون چندلایه}
{عرفان نصرتی}
{استاد آرش عبدی‌هجران‌دوست}
{}
{پروژه دوم}

\section*{}
\section*{مقدمه}
در این پروژه قصد داریم که با استفاده از کتابخانه آماده sklearn به پیاده سازی شبکه عضبی پرسپترون چند لایه بپردازیم. 
\section{سوال یک}
در سوال اول از ما خواسته شده است که توابعی با ورودی یک بعدی از ساده و سخت ایجاد کنیم و شبکه عصبی خود را روی هر یک از این تابع ها تست کنیم و تاثیر هر یک از عوامل ورودی را بر خروجی تابع مشاهده کنیم. در ابتدا سه تابع استفاده شده در این شبیه سازی من عبارت‌اند از $  y = x+4 $ که این تابع به عنوان تابع ساده استفاده شده است و یک تابع خطی از ورودی به خروجی است. تابع متوسط ، 
$  y = sin(\frac{x\pi}{5}) $
است که یک سینوسی با ساده است. و اما در آخر تابع سخت من برابر است با، $  y= sin(2\pi x) + sin(5\pi x) $ که این تابع برابر است با مجموع دو تابع سینوسی با فرکانس متفاوت.
\newline
حال در زیر به جزئیات پیاده‌سازی و خواسته های‌ مسئله از قبیل تاثیر هر کدام از پارامتر ها بر جواب مسئله می‌پردازیم.
\subsection{تابع آسان}

 \begin{figure}[H]
\includegraphics[width=0.8\textwidth]{../q1/code}
\centering
\end{figure}
همانطور که در بالا قابل مشاهده است، در ابتدا در خطوط ۱۵ و ۱۶ ما ورودی های شبکه عصبی خود را ایجاد می‌کنیم این ورودی ها به صورت پارامتریک هستند تا بعدا با تعویض آنها اثر آنها را بر شبکه عصبی پیدا کرد.
در ادامه این کار را برای نقاط تست هم انجام می‌دهیم تا مقدار واقعی نقاطی را که قرار است به وسیله شبکه عصبی تخمین بزنیم داشته باشیم. و سپس یک شبکه عصبی با تعداد تکرار محاسبات و تعداد لایه ها ایجاد می‌کنیم لازم به توضیح است از اینجای پروژه به بعد (10,10,30) این عبارت به این معنی است که لایه پنهان شبکه عصبی دارای ۳لایه است و تعداد نورون های این لایه ها به ترتیب برابر هستند با 10 و 10 و 30 یعنی جمعا  50 نورون داریم. حال در زیر تصاویر مختلف و تاثیرات پارامتر هارا می‌اوریم و آنها را بررسی می‌کنیم.
\newpage
\subsubsection{تعداد چرخه هاي شبکه براي تکمیل یادگیری}
\begin{figure}[!htb]
\minipage{0.32\textwidth}
  \includegraphics[width=\linewidth]{../q1/linear_(-50, 50)_(-300, 300)_100_(10, 10)}
  \caption{iter: 100}
\endminipage\hfill
\minipage{0.32\textwidth}
  \includegraphics[width=\linewidth]{../q1/linear_(-50, 50)_(-300, 300)_250_(10, 10)}
  \caption{iter: 250}
\endminipage\hfill
\minipage{0.32\textwidth}
  \includegraphics[width=\linewidth]{../q1/linear_(-50, 50)_(-300, 300)_500_(10, 10)}
  \caption{iter: 500}
\endminipage

\end{figure}

همانطور که در شکل ها قسمت قبل دیده می‌شود با افزایش iter تا دقت تخمین ما افزایش می‌یابد به صورتی که مشاهده می‌شود در زمانی که iter برابر ۱۰۰ است این مقدار برای یادگیری شبکه کافی نیست با اینکه کلا ۲۰ نورون در این شبکه عصبی وجود دارد اما تعداد ۱۰۰ مرتبه کافی نیست همین موضوع را می‌توان برای 250 مرتبه تکرار هم مشاهده کرد به صورتی که نتیجه بدست آمده بهتر از حالت ۱۰۰ است اما همچنان خطا بسیار بالاست اما با زیاد شدن تعداد ایتریشن ها یادگیری شبکه بهتر می‌شود اما بده بستانی که در این قسمت وجود دارد این است که مدت زمان بیشتری طول می‌کشد تا شبکه آموزش ببیند.

 \subsubsection{تعداد لایه هاي شبکه و تعداد نورون هاي هر لایه}
\begin{figure}[!htb]
\minipage{0.32\textwidth}
  \includegraphics[width=\linewidth]{../q1/linear_(-50, 50)_(-300, 300)_500_(10, 10)}
  \caption{10 10 }
\endminipage\hfill
\minipage{0.32\textwidth}
  \includegraphics[width=\linewidth]{../q1/linear_(-50, 50)_(-300, 300)_500_10}
  \caption{10}
\endminipage\hfill
\end{figure}

حال در بالا یک بار یک شبکه 10 لایه آمده است و بار دیگر همان مشخصات تنها با یک شبکه دولایه که هر کدام از لایه ها ۱۰ نورون دارد امده است. همانطور که دیده می‌شود با اینکه تابع ما یک تابع بسیار ساده است اما همانطور که انتظار می‌رود که با افزودن لایه‌های جدید تابع های با هزنیه های کمتری یادگرفته شوند به همین خاطر دیده می‌شود که خطای شبکه حاصل از دولایه کمتر از خطای شبکه حاصل از تک لایه است. اما تغییر خیلی محسوس نیست.
\newpage
\subsubsection{وسعت دامنه ورودي }
\begin{figure}[!htb]
\minipage{0.32\textwidth}
  \includegraphics[width=\linewidth]{../q1/linear_(-10, 10)_(-300, 300)_1000_10}
  \caption{بازه بادگیری از -10 تا 10}
\endminipage\hfill
\minipage{0.32\textwidth}
  \includegraphics[width=\linewidth]{../q1/linear_(-20, 20)_(-300, 300)_1000_10}
  \caption{بازه یادگیری از -20 تا 20}
\endminipage\hfill
\minipage{0.32\textwidth}
  \includegraphics[width=\linewidth]{../q1/linear_(-20, 20)_(-300, 300)_1000_10}
  \caption{بازه یادگیری از -30 تا 30}
\endminipage
\end{figure}

\begin{figure}[!htb]
\minipage{0.32\textwidth}
  \includegraphics[width=\linewidth]{../q1/linear_(-40, 40)_(-300, 300)_1000_10}
  \caption{بازه یادگیری از -40 تا 40}
\endminipage\hfill
\minipage{0.32\textwidth}
  \includegraphics[width=\linewidth]{../q1/linear_(-50, 50)_(-300, 300)_1000_10}
  \caption{بازه یادگیری از -50 تا 50}
\endminipage\hfill
\minipage{0.32\textwidth}
  \includegraphics[width=\linewidth]{../q1/linear_(-60, 60)_(-300, 300)_1000_10}
  \caption{بازه یادگیری از -60 تا 60}
\endminipage
\end{figure}
در این قسمت می‌خواهیم تاثیر بازه ورودی شبکه عصبی برای یادگیری را بررسی کنیم برای اینکه تنها طول بازه برای ما مطرح باشد تعداد نقاطی که در این بازه قرار دارد همواره ثابت و برابر ۱۰۰ است و تنها طول بازه تغییر می‌کند در این صورت همانطور که در بالا آمده است با افزایش طول بازه می‌بینیم که خطا کمتر می‌شود و این بیانگر این است که اگر تعداد ثابتی نقطه داشته باشیم بهتر است در یک بازه به نسبت بزرگی از تابع این نقاط پراکنده باشند تا شبکه بتواند بهتر یاد بگیرد.




\newpage
\subsubsection{تعداد نقاط ورودی}



\begin{figure}[!htb]
\minipage{0.32\textwidth}
  \includegraphics[width=\linewidth]{../q1/linear_10}
  \caption{points: 10}
\endminipage\hfill
\minipage{0.32\textwidth}
  \includegraphics[width=\linewidth]{../q1/linear_20}
  \caption{points: 20}
\endminipage\hfill
\minipage{0.32\textwidth}
  \includegraphics[width=\linewidth]{../q1/linear_30}
  \caption{points: 30}
\endminipage
\end{figure}

\begin{figure}[!htb]
\minipage{0.48\textwidth}
  \includegraphics[width=\linewidth]{../q1/linear_100}
  \caption{points: 100}
\endminipage\hfill
\minipage{0.48\textwidth}
  \includegraphics[width=\linewidth]{../q1/linear_200}
  \caption{points:: 200}
\endminipage\hfill
\end{figure}
در نمودار های بالا به بررسی اثر تعداد نقاط بر دقت ما آمده است و دیده می‌شود اثر این بخش بسیار کمتر از بخش وسعت دامنه ورودی است یعنی با افزایش تعداد نقاط ورودی خطای تابع به صورت قابل ملاحظه کاهش نمی‌یابد.

\newpage
\subsection{تابع متوسط}
کد نوشته شده برای این قسمت مانند قسمت قبل است تنها فرق آن این است که تابع پیچیده تر است و در زیر می‌خوایم اثر پارامتر هارا بر این تابع بررسی کنیم.

\subsubsection{تعداد چرخه هاي شبکه براي تکمیل یادگیری}

\begin{figure}[!htb]
\minipage{0.32\textwidth}
  \includegraphics[width=\linewidth]{../q1/sin_(-50, 50)_(-150, 150)_100_(10, 30, 10)}
  \caption{iter: 100}
\endminipage\hfill
\minipage{0.32\textwidth}
  \includegraphics[width=\linewidth]{../q1/sin_(-50, 50)_(-150, 150)_200_(10, 30, 10)}
  \caption{iter: 200}
\endminipage\hfill
\minipage{0.32\textwidth}
  \includegraphics[width=\linewidth]{../q1/sin_(-50, 50)_(-150, 150)_500_(10, 30, 10)}
  \caption{iter: 500}
\endminipage
\end{figure}
همانطور که مانند قسمت قبل انتظار می‌رفت با افزایش تعداد چرخه ما به تابع نهایی نزدیک تر می‌شویم ولی در اینجا خطای گزارش شده میزان درستی نیست چون تابع در بازه ای که آموزش ندیده دارد شکل بهتری پیدا می‌کند ام خطا این مشکل را دارد که چون سینوسی است و اگر یک خط از وسط آن رد شود خطای آن کمتر از حالت پیش است پس نمی‌توان MSE را معیار درستی برای سنجش قرار داد.


 \subsubsection{تعداد لایه هاي شبکه و تعداد نورون هاي هر لایه}
\begin{figure}[!htb]
\minipage{0.32\textwidth}
  \includegraphics[width=\linewidth]{../q1/sin_(-50, 50)_(-150, 150)_500_10}
  \caption{10}
\endminipage\hfill
\minipage{0.32\textwidth}
  \includegraphics[width=\linewidth]{../q1/sin_(-50, 50)_(-150, 150)_500_(10, 30)}
  \caption{(10, 30)}
\endminipage\hfill
\minipage{0.32\textwidth}
  \includegraphics[width=\linewidth]{../q1/sin_(-50, 50)_(-150, 150)_500_(10, 30, 30)}
  \caption{(10, 30, 30)}
\endminipage\hfill
\end{figure}
همانطور که دیده میشود مانند حالت قبلی با افزایش تعداد نورون ها و لایه های شبکه دقت شبکه عصبی ما افزایش می‌یابد.



\newpage
\subsubsection{وسعت دامنه ورودي }
\begin{figure}[!htb]
\minipage{0.32\textwidth}
  \includegraphics[width=\linewidth]{../q1/sin_(-10, 10)_(-150, 150)_500_(10, 30, 30, 10)}
  \caption{بازه بادگیری از -10 تا 10}
\endminipage\hfill
\minipage{0.32\textwidth}
  \includegraphics[width=\linewidth]{../q1/sin_(-20, 20)_(-150, 150)_500_(10, 30, 30, 10)}
  \caption{بازه یادگیری از -20 تا 20}
\endminipage\hfill
\minipage{0.32\textwidth}
  \includegraphics[width=\linewidth]{../q1/sin_(-50, 50)_(-150, 150)_500_(10, 30, 30, 10)}
  \caption{بازه یادگیری از -30 تا 30}
\endminipage
\end{figure}
در مورد با اتفاق جالب ایت است که در یک تابع که دارای پیچدگی است اگر دامنه به اندازه کافی بزرگ نباشد باعث می‌شود که در این حالت ما نتوانیم تخمین بسیار بدی از تابع بزنیم مانند بالا که تابع در ورودی 10تا -10 به صورت یک تابع خطی تقریب زده شده است. و با افزایش وسعت دامنه ورودی افزایش می‌یابد


\subsubsection{تعداد نقاط ورودی}



\begin{figure}[!htb]
\minipage{0.32\textwidth}
  \includegraphics[width=\linewidth]{../q1/sin_10}
  \caption{points: 10}
\endminipage\hfill
\minipage{0.32\textwidth}
  \includegraphics[width=\linewidth]{../q1/sin_50}
  \caption{points: 20}
\endminipage\hfill
\minipage{0.32\textwidth}
  \includegraphics[width=\linewidth]{../q1/sin_200}
  \caption{points: 30}
\endminipage
\end{figure}

\begin{figure}[!htb]
\minipage{0.32\textwidth}
  \includegraphics[width=\linewidth]{../q1/sin_500}
  \caption{points: 10}
\endminipage\hfill
\minipage{0.32\textwidth}
  \includegraphics[width=\linewidth]{../q1/sin_500}
  \caption{points: 20}
\endminipage\hfill
\minipage{0.32\textwidth}
  \includegraphics[width=\linewidth]{../q1/sin_1000}
  \caption{points: 30}
\endminipage
\end{figure}
همانطور که انتظار می‌رفت با افزایش تعداد نقاط دقت تخمین ما هم افزایش می‌یابد همچنین با توجه به اینکه در تابع سینوسی ما تغییرات شدید تری نسبت به توابع خطی داریم دیده می‌شود که این میزان تاثیر تعداد نقاط ورودی بر خطای تابع بسیار بیشتر از حالت خطی است اما در این حالت ما نیز یک حد داریم زیرا از ۵۰۰ به ۱۰۰۰ ما تغییر محسوسی نکرده ایم.







%%%%%%%%%%%%%%%%%%%%hard%%%%%%%%%%%%%%%%%%%%%%%%%

\newpage
\subsection{تابع سخت}
کد نوشته شده برای این قسمت مانند قسمت قبل است تنها فرق آن این است که تابع پیچیده تر است و در زیر می‌خوایم اثر پارامتر هارا بر این تابع بررسی کنیم.

\subsubsection{تعداد چرخه هاي شبکه براي تکمیل یادگیری}

\begin{figure}[!htb]
\minipage{0.32\textwidth}
  \includegraphics[width=\linewidth]{../q1/exponential_(-2, 2)_(-4, 4)_100_(20, 40, 50, 30)}
  \caption{iter: 100}
\endminipage\hfill
\minipage{0.32\textwidth}
  \includegraphics[width=\linewidth]{../q1/exponential_(-2, 2)_(-4, 4)_200_(20, 40, 50, 30)}
  \caption{iter: 200}
\endminipage\hfill
\minipage{0.32\textwidth}
  \includegraphics[width=\linewidth]{../q1/exponential_(-2, 2)_(-4, 4)_400_(20, 40, 50, 30)}
  \caption{iter: 400}
\endminipage
\end{figure}
در این قسمت با افزایش تعداد چرخه ها دقت ما افزایش می‌یاد اما از ۲۰۰ به بعد این مفدار تغییر نمی‌کند زیرا شبکه به یک مقدار ثابت همگرا شده است و تعداد دفعات تکرار بیش از این تعداد فایده ندارد.


 \subsubsection{تعداد لایه هاي شبکه و تعداد نورون هاي هر لایه}
\begin{figure}[!htb]
\minipage{0.32\textwidth}
  \includegraphics[width=\linewidth]{../q1/exponential_(-2, 2)_(-4, 4)_400_(20, 10)}
  \caption{10}
\endminipage\hfill
\minipage{0.32\textwidth}
  \includegraphics[width=\linewidth]{../q1/exponential_(-2, 2)_(-4, 4)_400_(20, 10, 30)}
  \caption{(10, 30)}
\endminipage\hfill
\minipage{0.32\textwidth}
  \includegraphics[width=\linewidth]{../q1/exponential_(-2, 2)_(-4, 4)_400_(20, 40, 50, 30)}
  \caption{(10, 30, 30)}
\endminipage\hfill
\end{figure}
ما انتظار داشتیم که در توابع پیجیده تر تعداد نورون ها و لایه ها برای ما مهم تر باشند و همین طور هم شد و در بلا دیهد می‌شود که تاثیرات افزودن شبکه چگونه است در قسمت اول ما یک شبکه دو لایه با ۲۰ و ۱۰ نورون داریم در قسمت بعدی یک لایه با  ۲۰ و ۱۰ و ۳۰ نورون و در اخری با ۲۰ و ۴۰ و ۵۰ و ۳۰ داریم.


\newpage
\subsubsection{وسعت دامنه ورودي }
\begin{figure}[!htb]
\minipage{0.32\textwidth}
  \includegraphics[width=\linewidth]{../q1/exponential_(-2, 2)_(-4, 4)_400_(20, 40, 50, 30)}
  \caption{بازه بادگیری از -2 تا 2}
\endminipage\hfill
\minipage{0.32\textwidth}
  \includegraphics[width=\linewidth]{../q1/exponential_(-3, 3)_(-4, 4)_400_(20, 40, 50, 30)}
  \caption{بازه یادگیری از -3 تا 3}
\endminipage\hfill
\minipage{0.32\textwidth}
  \includegraphics[width=\linewidth]{../q1/exponential_(-4, 4)_(-4, 4)_400_(20, 40, 50, 30)}
  \caption{بازه یادگیری از -4 تا 4}
\endminipage
\end{figure}
در مورد این تابع در شکل های بالاتر مشاهده شد که قسمت هایی که در دامنه یادگیری نیستند به درستی تشخیص داده نمی‌شدند و این نشان دهنده اهمیت وسعت دامنه ورودی در تابع پیچیده تر است به صورتی که در صورتی که شما از کل فضا تعداد نمونه داشته باشید بسیار بهتر از حالتی است که از قسمتی کوچک مقدار زیادی نمونه دارید.

\subsubsection{تعداد نقاط ورودی}



\begin{figure}[!htb]
\minipage{0.32\textwidth}
  \includegraphics[width=\linewidth]{../q1/exponential_1000}
  \caption{points: 1000}
\endminipage\hfill
\minipage{0.32\textwidth}
  \includegraphics[width=\linewidth]{../q1/exponential_2000}
  \caption{points: 2000}
\endminipage\hfill
\minipage{0.32\textwidth}
  \includegraphics[width=\linewidth]{../q1/exponential_3000}
  \caption{points: 3000}
\endminipage
\end{figure}

\begin{figure}[!htb]
\minipage{0.32\textwidth}
  \includegraphics[width=\linewidth]{../q1/exponential_5000}
  \caption{points: 5000}
\endminipage\hfill
\minipage{0.32\textwidth}
  \includegraphics[width=\linewidth]{../q1/exponential_6000}
  \caption{points: 6000}
\endminipage\hfill
\minipage{0.32\textwidth}
  \includegraphics[width=\linewidth]{../q1/exponential_7000}
  \caption{points: 7000}
\endminipage
\end{figure}
همانطور که انتظار می‌رفت با افزایش تعداد نقاط دقت تخمین ما هم افزایش می‌یابد همچنین با توجه به اینکه در تابع سینوسی ما تغییرات شدید تری نسبت به توابع خطی داریم دیده می‌شود که این میزان تاثیر تعداد نقاط ورودی بر خطای تابع بسیار بیشتر از حالت خطی است و بر خلاف حالت سینوسی قبلی با افزایش تعداد نقاط همواره دقت افزایش می‌یابد زیرا این تابع تغیرات شدید تری دارد




\newpage
\section{سوال دوم}

\subsection{آسان}

\begin{figure}[!htb]
\minipage{0.48\textwidth}
  \includegraphics[width=\linewidth]{../q2/linear_small}
  \caption{بدون نویز}
\endminipage\hfill
\minipage{0.48\textwidth}
  \includegraphics[width=\linewidth]{../q2/linear_1small}
  \caption{مقدار کم}
\endminipage\hfill
\end{figure}


\begin{figure}[!htb]
\minipage{0.48\textwidth}
  \includegraphics[width=\linewidth]{../q2/linear_10small}
  \caption{مقدار متوسط}
\endminipage\hfill
\minipage{0.48\textwidth}
  \includegraphics[width=\linewidth]{../q2/linear_100small}
  \caption{مقدار زیاد}
\endminipage\hfill
\end{figure}
\newpage
\subsection{متوسط}

\begin{figure}[!htb]
\minipage{0.48\textwidth}
  \includegraphics[width=\linewidth]{../q2/sin_small}
  \caption{بدون نویز}
\endminipage\hfill
\minipage{0.48\textwidth}
  \includegraphics[width=\linewidth]{../q2/sin_1small}
  \caption{مقدار کم}
\endminipage\hfill
\end{figure}


\begin{figure}[!htb]
\minipage{0.48\textwidth}
  \includegraphics[width=\linewidth]{../q2/sin_10small}
  \caption{مقدار متوسط}
\endminipage\hfill
\minipage{0.48\textwidth}
  \includegraphics[width=\linewidth]{../q2/sin_100small}
  \caption{مقدار زیاد}
\endminipage\hfill
\end{figure}

\newpage
\subsection{سخت}

\begin{figure}[!htb]
\minipage{0.48\textwidth}
  \includegraphics[width=\linewidth]{../q2/exponential_small}
  \caption{بدون نویز}
\endminipage\hfill
\minipage{0.48\textwidth}
  \includegraphics[width=\linewidth]{../q2/exponential_1small}
  \caption{مقدار کم}
\endminipage\hfill
\end{figure}


\begin{figure}[!htb]
\minipage{0.48\textwidth}
  \includegraphics[width=\linewidth]{../q2/exponential_5small}
  \caption{مقدار متوسط}
\endminipage\hfill
\minipage{0.48\textwidth}
  \includegraphics[width=\linewidth]{../q2/exponential_10small}
  \caption{مقدار زیاد}
\endminipage\hfill
\end{figure}

برای نویز من در تصاویر بالا ۴ حالت بدون نویز با نویز کم و با نویز متوسط و با نویز زیاد را آورده ام. در حالتی که نویز کم باشد همه مدل ها به درستی عمل می‌کنند بدین صورت که خطا ها افزایش می‌یابد اما این افزایش خطا به میزانی نیست که شبکه ها را دچار مشکل کند و یا خروجی ها غیرقابل قبول شوند. با افزایش نویز تابع هایی که ضابطه سخت تری داشته اند به سرعت بیشتری خراب می‌شوند زیرا تابع‌های پیچیده تر دارای تغیرات ناگهانی در ارتفاع هستند و این باعث می‌شود نوییز هایی که در مقابل دامنه آنها قابل صرف تظر تباشد بتواند آن‌ها را به سرعت خراب کند. اما در مورد تابع خطی می‌بینیم که به اینکه نویز زیاد عملا ۱۰ برابر دامنه تابع نویز ایجاد کرده است باز هم با تقریب بهتری نزدیک جواب مانده است پس نتیجه می‌گیریم که نویز بر توابع پیچیده‌تر اثر شدید تری دارند.

\newpage
\section{سوال سوم}
در قسمت سوم می‌خواهیم توابعی با ابعاد بالاتر را مورد بررسی قرار دهیم برای این کار از سه تابع ۳ بعدی باز ضابطه های آسان، متوسط و سخت مانند سوال یک استفاده می‌کنیم تابع های ما به صورت $ z = f(x,y) $  هستند که در زیر ببرسی هر یک میپردازیم و بهترین تابع را برای هر کدام پیدا می‌کنیم.

\subsection{ضابطه آسان}
ضابطه‌ی این تابع برابر بود با $ z = x +y $ که یک عبارت خطی ساده است برای آموزش این تابع از یک شبکه دو لایه استفاده کردیم. اولین چالشی که با آن برخورد کردم این بود که این تابع نسبت به تابع های دو بعدی بسیار زمان بیشتری برای آموزش و تست مدل ها می‌برد به همین دلیل نمی‌شد پارامتر های زیادی را مانند آنچه در قسمت های قبل استفاده شد تست کرد زیرا برای تست کردن شبکه دولایه که از ۵ تا ۴۰ نورون در لایه اول و ۵ تا ۴۰ نورون در لایه دوم تشکیل شده بودند چیزی حدود نیم ساعت زمان می‌برد و به علت سختی این توابع خدس زدن مقدار درست و دیدن شهودی آن کمکی به موضوع نمی‌کرد.
در ابتدا در زیر مقدایر حساب شده چند مقدار به صورت تستی برای بدست آوردن رنج شبکه را آورده‌ام سپس در قسمت بعدی روی تمام حالات ممکن  در بازه پیدا شده حلقه زده شده است که در زیر نتایج آنها را می‌اورم.
\begin{figure}[!htb]
\minipage{0.48\textwidth}
  \includegraphics[width=\linewidth]{../q3/Easy}
  \caption{ امقدار دستی اول}
\endminipage\hfill
\minipage{0.48\textwidth}
  \includegraphics[width=\linewidth]{../q3/Easy1}
  \caption{مقدار دستی دوم}
\endminipage\hfill
\end{figure}

\begin{figure}[!htb]
\minipage{0.48\textwidth}
  \includegraphics[width=\linewidth]{../q3/Easy2}
  \caption{مقدار  دستی سوم}
\endminipage\hfill
\end{figure}
\newpage
همانطور که مشاهده می‌شود در زمانی که دو لایه ۱۰ و ۱۰ داریم حالت بهتری داریم و خطا کمتر است به همین دلیل تمام حالات در بازه شبکه های دولایه را بررسی می‌کنیم تا به بهترین جواب برسیم.( یه جوری سرچ محلی زدم)
پس همانطور که در شکل دیده می‌شود بهترین شبکه به ازای شبکه با -- در لایه اول و -- در لایه دوم بدست آمده است. و خطای MSE آن عبارت است از : 68.73


\begin{figure}[!htb]
  \includegraphics[width=\linewidth]{../q3/Easy3}
  \caption{بهترین شبکه}
\end{figure}

\newpage
\subsection{ضابطه متوسط}
در این قسمت برای سخت تر کردن تابع از یک تابع ۴ بعدی استفاده کرده ام که ضابطه‌ي آن به شکل $        f(x,y) = (x+y, x-y)  $ است و  تمام کار هایی که در قسمت قبلی به آن اشاره شد در این قسمت نیز انجام شده است با این تفاوت که در این قسمت با توجه به سخت تر بودم ضابطه یک شبکه سه ساده نیز تست شد که در آن خطا ها کمتر بود پس از آن تمام مقدایر بین ۱۰ تا ۲۵ نورون برای شبکه سه لایه مورد بررسی قرار گرفت که نتایج به شرح زیر است.


\begin{figure}[!htb]
\minipage{0.48\textwidth}
  \includegraphics[width=\linewidth]{../q3/Medium}
  \caption{ امقدار دستی اول}
\endminipage\hfill
\minipage{0.48\textwidth}
  \includegraphics[width=\linewidth]{../q3/Medium1}
  \caption{مقدار دستی دوم}
\endminipage\hfill
\end{figure}

\begin{figure}[!htb]
\minipage{0.48\textwidth}
  \includegraphics[width=\linewidth]{../q3/Medium2}
  \caption{ امقدار دستی سوم}
\endminipage\hfill
\end{figure}
\newpage
و پس از انجام این مراحل همانطور که در بالا گفته شد تمام توابع سه لایه که بین ۱۰ تا ۲۵ نورون داشتند چک شد.

\begin{figure}[!htb]
\minipage{0.48\textwidth}
  \includegraphics[width=\linewidth]{../q3/Medium3}
  \caption{ تمامی شبکه ‌های با تعداد نورون و لایه مشخص شده }
\endminipage\hfill
\minipage{0.48\textwidth}
  \includegraphics[width=\linewidth]{../q3/Medium4}
  \caption{تمامی شبکه ‌های با تعداد نورون و لایه مشخص شده}
\endminipage\hfill
\end{figure}
همانطور که دیده می‌شود بهترین شبکه متعلق به شبکه ای با ۱۶، ۱۶ و ۱۰ لایه نورون به ترتیب در لایه اول، دوم و سوم است.

\newpage
\subsection{ضابطه سخت}
در این قسمت برای سخت تر کردن تابع از یک تابع ۴ بعدی استفاده کرده ام که ضابطه‌ي آن به شکل $        f(x,y,) =( (x+y)^3, (x-y)^2)  $ است و  تمام کار هایی که در قسمت قبلی به آن اشاره شد. در زیر عکس های‌ آن آورده شده است.

\begin{figure}[!htb]
\minipage{0.48\textwidth}
  \includegraphics[width=\linewidth]{../q3/Hard}
  \caption{ امقدار دستی اول}
\endminipage\hfill
\minipage{0.48\textwidth}
  \includegraphics[width=\linewidth]{../q3/Hard1}
  \caption{مقدار دستی دوم}
\endminipage\hfill
\end{figure}

\begin{figure}[!htb]
\minipage{0.48\textwidth}
  \includegraphics[width=\linewidth]{../q3/Hard3}
  \caption{ امقدار دستی سوم}
\endminipage\hfill
\end{figure}
\newpage
و پس از انجام این مراحل همانطور که در بالا گفته شد تمام توابع 6 لایه که بین 100 تا 100 نورون داشتند چک شد.

\begin{figure}[H]
\begin{center}

  \includegraphics[width=\linewidth]{../q3/Hard4}
  \center
  \caption{ تمامی شبکه ‌های با تعداد نورون و لایه مشخص شده }
  \center
\end{center}
\end{figure}

همانطور که دیده می‌شود خطا بسیار بالاست و احتمالا مدل به درسی آموزش داده نشده است اما از آنجایی که خروحی ما با تفاضل ورودی ها رابطه توان ۲ و با جمع ان ها رابطه توان ۳ دارد به همین دلیل باعث شده است که در این روش اگر خطا داشته باشیم خطا به صورت توانی زیاد شود و این باعث این میزان از خطا شده است.
\newpage
\section{سوال چهارم}
در این سوال از ما خواسته شده بود که یک شکل در کاغذ بکشیم و مقدار بالا یا پایین آن را به عنوان یک تابع در نظر بگیریم در زیر شکل کشیده شده آورده شده است و سپس مقدار تخمین زده آن نیز در قسمت بعدی آورده شده است.
\begin{figure}[H]
\begin{center}

  \includegraphics[width=\linewidth]{../q4/sign}
  \center
  \caption{ مقدار کشیده شده در paint}
\end{center}
\end{figure}
در زیر ابندا مدل هایی که به اشتباه ترین شده اند آمده است و در آخر مدلی که به درستی آموزش داده شده بود آمده است.

\begin{figure}[!htb]
\minipage{0.48\textwidth}
  \includegraphics[width=\linewidth]{../q4/linear_1-71000_[10, 9]}
  \caption{  اولین تلاش با شبکه (10,9) }
\endminipage\hfill
\minipage{0.48\textwidth}
  \includegraphics[width=\linewidth]{../q3/linear_1-71000_[10, 8]}
  \caption{دومین تلاش با شبکه (10,8) }
\endminipage\hfill
\end{figure}

\begin{figure}[!htb]
\minipage{0.48\textwidth}
  \includegraphics[width=\linewidth]{../q4/linear_1-71000_[9, 9]}
  \caption{  سومین تلاش با شبکه (9,9) }
\endminipage\hfill
\minipage{0.48\textwidth}
  \includegraphics[width=\linewidth]{../q3/linear_1-71000_[10, 10, 10]}
  \caption{چهارمین تلاش با شبکه (10,10,10 }
\endminipage\hfill
\end{figure}

\begin{figure}[!htb]
\minipage{0.48\textwidth}
  \includegraphics[width=\linewidth]{../q4/linear_1-71000_[10, 10]}
  \caption{  پنجمین تلاش با شبکه (10,10) }
\endminipage\hfill
\minipage{0.48\textwidth}
  \includegraphics[width=\linewidth]{../q3/linear_1-71000_[10, 11]}
  \caption{ششمین تلاش با شبکه (10,9 }
\endminipage\hfill
\end{figure}


\begin{figure}[!htb]
\minipage{0.48\textwidth}
  \includegraphics[width=\linewidth]{../q4/linear_1-71000_[12, 12]}
  \caption{  هفتمین تلاش با شبکه (12,12) }
\endminipage\hfill
\minipage{0.48\textwidth}
  \includegraphics[width=\linewidth]{../q3/linear_1-71000_[10, 20]}
  \caption{هشتمین تلاش با شبکه (10,20 }
\endminipage\hfill
\end{figure}


\begin{figure}[H]
\begin{center}
  \includegraphics[width=\linewidth]{../q4/linear_1-71000_[10, 10]}
  \center
  \caption{ بهترین مقدار تخمین شده شده}
\end{center}
\end{figure}

\end{document}
